\documentclass[12pt]{article}
\usepackage{amssymb}
\usepackage{amsmath}
\title{Mathematical Analysis Homework 4: Solutions}
\author{Cory Nezin}
\begin{document}
\maketitle
\begin{enumerate}
\item Prove that every interval $I$ is one of the nine types:
\begin{enumerate}
\item $I = [a,b]$
\item $I = (a,b)$
\item $I = (a,b]$
\item $I = [a,b)$
\item $I = (-\infty, b]$
\item $I = (-\infty, b)$
\item $I = (a,\infty)$
\item $I = [a,\infty)$
\item $I = (-\infty,\infty)$
\end{enumerate}
Proof:\\
Since the cardinality of an interval must be at least $2$, the interval is not empty and not singleton.\\
We seperate the proof into two cases: $I$ is bounded (1) and $I$ is not bounded (2).\\
We further seperate case 1 into four cases: 
\begin{enumerate}
\item $\sup I \in I \inf I \in I$
\item $\inf I \notin I, \sup I \notin I$
\item $\sup I \notin I, \inf I \in I$
\item $\sup I \in I, \inf I \notin I$
\end{enumerate}
Let $b = \sup I, a = \inf I$\\
\begin{enumerate}
\item Case 1.a:\\
Since $a\in I$ and $b\in I$, by definition of interval, $\forall x\in \mathbb{R} \text{ where } a\leq x\leq b \rightarrow x\in I$ so $[a,b] \subseteq I$.  By defintion of $\sup$ and $\inf$, $x\in I \rightarrow a\leq x\leq b \rightarrow x\in [a,b]$.  Therefore $I = [a,b]$

\item Case 1.b:\\
Since $|I| \geq 2, \exists x,y\in I \text{ such that } a<x<y<b$\\
$\forall t\in (a,b), a<t<b$.  \\
There are one of three possibilities:\\
If $x<t<y, t\in I$ by definition of an interval.\\
If $a < t < x, \text{ since $\inf I < t$, t is not a lower bound of I } \rightarrow \exists z\in I \text{ such that } z < t$.  Since $z<t<x$, $t\in I$\\
If $y < t < b, \text{ since $\sup I > t$, t is not a upper bound of I } \rightarrow \exists z\in I \text{ such that } t < z$.  Since $y<t<z$, $t\in I$\\
so $(a,b) \subseteq I$\\
Since $\sup I = b$ and $\inf I = a$, $\forall t\in I, a<t<b \rightarrow t\in (a,b) \rightarrow I \subseteq (a,b)$ and thus $I=(a,b)$

\item Case 1.c:\\
Since $|I| \geq 2, \exists x,y\in I \text{ such that } a<x<y<b$\\
$\forall t\in (a,b], a<t<b$.  \\
There are one of four possibilities:\\
If $x<t<y, t\in I$ by definition of an interval.\\
If $a < t < x, \text{ since $\inf I < t$, t is not a lower bound of I } \rightarrow \exists z\in I \text{ such that } z < t$.  Since $z<t<x$, $t\in I$\\
If $y < t < b, \text{ since $\sup I > t$, t is not a upper bound of I } \rightarrow \exists z\in I \text{ such that } t < z$.  Since $y<t<z$, $t\in I$\\
If $t=b$, then since $b = \sup I \in I$, $t\in I$
so $(a,b) \subseteq I$\\
Since $\sup I = b$ and $\inf I = a$, $\forall t\in I, a<t<b \rightarrow t\in (a,b) \rightarrow I \subseteq (a,b)$ and thus $I=(a,b)$
\item Case 1.d:\\
Since $|I| \geq 2, \exists x,y\in I \text{ such that } a<x<y<b$\\
$\forall t\in (a,b], a<t<b$.  \\
There are one of four possibilities:\\
If $x<t<y, t\in I$ by definition of an interval.\\
If $a < t < x, \text{ since $\inf I < t$, t is not a lower bound of I } \rightarrow \exists z\in I \text{ such that } z < t$.  Since $z<t<x$, $t\in I$\\
If $y < t < b, \text{ since $\sup I > t$, t is not a upper bound of I } \rightarrow \exists z\in I \text{ such that } t < z$.  Since $y<t<z$, $t\in I$\\
If $a=t$, then since $a = \inf I \in I$, $t\in I$
so $(a,b) \subseteq I$\\
Since $\sup I = b$ and $\inf I = a$, $\forall t\in I, a<t<b \rightarrow t\in (a,b) \rightarrow I \subseteq (a,b)$ and thus $I=(a,b)$\\
\end{enumerate}
We now suppose I is unbounded and consider the 5 possible cases:
\begin{enumerate}
\item I is bounded above, $b = \sup I \in I$
\item I is bounded above, $b = \sup I \notin I$
\item I is bounded below, $a = \inf I \in I$
\item I is bounded below, $a = \inf I \notin I$
\item I is bounded neither above nor below.
\end{enumerate}
\begin{enumerate}
\item $(-\infty,b] = \{t\in\mathbb{R}, t \leq b\}$\\
Since $I$ is not bounded below, $\forall a\in \mathbb{R}, \exists x\in I \text{ such that } x < a$\\
Suppose $t\in (-\infty,b]$, then either:\\
$t < x \rightarrow \exists z\in I \text{ such that } z < t \rightarrow z<t<x \rightarrow t\in I$ or\\
$t = x \rightarrow t\in I$ or\\
$t > x \rightarrow x < t < b \rightarrow t\in I$\\
So $(-\infty,b] \subseteq I$.\\
Since $\sup I = b, \forall t\in I, t \leq b \rightarrow t \in (-\infty,b] \rightarrow I\subseteq (-\infty,b]$\\
So $I = (-\infty,b]$
\item $(-\infty,b) = \{t\in\mathbb{R},t<b\}$\\
Since $I$ is not bounded below, $\forall a\in \mathbb{R}, \exists x\in I \text{ such that } x < a$\\
Suppose $t\in (-\infty,b)$, then either:\\
$t < x \rightarrow \exists z\in I \text{ such that } z < t \rightarrow z<t<x \rightarrow t\in I$ or\\
$t = x \rightarrow t\in I$ or\\
$t > x \rightarrow x < t < b$.  Since t can not be an upper bound for I, $\exists y\in I$ such that $y > t$ so $x < t < y$ and $t\in I$\\
So $(-\infty,b] \subseteq I$.\\
Since $\sup I = b, \forall t\in I, t < b \rightarrow t \in (-\infty,b) \rightarrow I\subseteq (-\infty,b)$\\
So $I = (-\infty,b)$
\item $[a,\infty) = \{t\in\mathbb{R}, t \geq a\}$\\

Since $I$ is not bounded above, $\forall b\in \mathbb{R}, \exists x\in I \text{ such that } x > b$\\
Suppose $t\in [a,\infty)$, then either:\\
$x < t \rightarrow \exists z\in I \text{ such that } z > t \rightarrow x<t<z \rightarrow t\in I$ or\\
$x = t \rightarrow t\in I$ or\\
$x > t \rightarrow a < t < x \rightarrow t\in I$\\
So $[a,\infty) \subseteq I$.\\
Since $\inf I = a, \forall t\in I, t > a \rightarrow t \in [a,\infty] \rightarrow I\subseteq [a,\infty)$\\
So $I = [a,\infty)$

\item $(a,\infty) = \{t\in\mathbb{R}, t > a\}$\\
Since $I$ is not bounded above, $\forall b\in \mathbb{R}, \exists x\in I \text{ such that } x > b$\\
Suppose $t\in (a,\infty)$, then either:\\
$x < t \rightarrow \exists z\in I \text{ such that } z > t \rightarrow x<t<z \rightarrow t\in I$ or\\
$x = t \rightarrow t\in I$ or\\
$x > t \rightarrow a < t < x$.  Since $t$ is not a lower bound, $\exists z\in I$ such that $z<t<x$ so $t\in I$.\\
So $(a,\infty) \subseteq I$.\\
Since $\inf I = a, \forall t\in I, t > a \rightarrow t \in (a,\infty) \rightarrow I\subseteq (a,\infty)$\\
So $I = (a,\infty)$
\item $(-\infty,\infty) = \mathbb{R}$\\
Since $I$ is bounded neither above nor below, $\forall a \in \mathbb{R}, \exists x\in I \text{ such that } x < a$ and $\forall b \in \mathbb{R}, \exists y\in I \text{ such that } b < y$.
Thus, $\forall t\in \mathbb{R}, \exists x,y\in I \text{ such that } x<t<y \rightarrow t\in I$\\
So $\mathbb{R} \subseteq I$.\\
By definition of intervals, $I \subseteq \mathbb{R}$.\\
So $I = (-\infty,\infty)$\\
Thus the nine intervals listed are the only types.
$\blacksquare$
\end{enumerate}
\item Let $\{I_\alpha\}_{\alpha\in A}$ be a collection of intervals in $\mathbb{R}$ where $A\neq \emptyset$.  Prove that $I = \bigcap_{\alpha\in A} I_{\alpha}$ is one of the following types:
\begin{enumerate}
\item $\emptyset$
\item $\{a\}$
\item An interval
\end{enumerate}
Proof:\\
In the case where the intersection contains no elements (e.g. $(-1,0)$ and $(0,1)$) we have case (a).\\
In the case where the intersection contains one element (e.g. $(-1,0]$ and $[0,1)$) we have case (b).\\
Now we examine the case where the intersection has two or more elements.\\
The first requirement of an interval is clearly satisfied.  Suppose the two elements in the intersection are $a$ and $b$ and $a<b$.\\
Since $a,b\in I_\alpha \text{ }\forall \alpha$, if $\exists t\in\mathbb{R}$ such that $a<t<b, t\in I_\alpha$.  So since $t$ is in every interval, it must be in the intersection of every interval.  Therefore the intersection satisfies both properties of intervals and it is an interval. $\blacksquare$

\end{enumerate}
\end{document}
